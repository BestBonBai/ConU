\newpage
\section{P and V as a non-critical section}
\subsection{Problems:}
Generally, both P and V operation must be implemented as a critical section. Are there any cases
when any of these two operations can safely be implemented as a non-critical section? If yes,
demonstrate through an example when/how this can be done without creating any violations. If
no, explain why these operations must always be implemented as critical sections. 
\subsection{Answer:}

These operations can be implemented as non critical sections.
For example, there are three process have to perform addition function on any two numbers. But the addition function is present as a function between P() and V() i.e if one process is using it the others cannot, So when a single process is using the addition function all other have to wait as per the P() function. It could have been possible that all the three process uses the addition function at once if it was not placed inside a P() and V() functions. Here the addition function is for local variables which is different for different process but if it had been global variables that would have been same for all the processes, it comes under the case of shared resource then the addition function should be placed between P() and V() function
