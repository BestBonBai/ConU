\newpage
\section{Swapping/relocation system}
\subsection{Problem:}
In a swapping/relocation system, the values assigned to the <base, limit> register pair prevent
one user process from writing into the address space of another user process. However, these
assignment operations are themselves privileged instructions that can only be executed in kernel
models
ode.
 Is it conceivable that some operating-system processes might have the entire main memory
 as their address space? If this is possible, is it necessarily a bad thing? Explain.

 \subsection{Answer:}
Usually, the operating system also resides in the low memory partition of the main memory. Other user processes use the high memory partition. It is possible that some operating-system process might have the enitre memory as their address space. Following are the implications -
\begin{itemize}
    \item slows down other processes that are waiting for the main memory.
    \item New processes are queued in the Secondary Storage. Resulting in Low Disk Space
    \item Processes do not interact with each other. Interaction might be necessary for inter process communication.
\end{itemize}
