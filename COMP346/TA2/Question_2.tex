\newpage
\section{Light-weight processes}
\subsection{Problem:}
\begin{itemize}
    \item Why threads are referred to as “light-weight” processes? 
    \item What resources are used when a thread is created?
    \item How do they differ from those used when a process is created?
\end{itemize}

\subsection{Answer:}
\begin{itemize}
    \item Threads are sometimes called lightweight processes because they have their own stack but can access shared data.
    \item Threads are smaller than processes, so they needs less resources. Threads allocate a small data structure to hold a register set, stack, and priority. A process allocates a PCB, which is a rather large data structure. Context switching with a thread is much quicker than a process.
\end{itemize}
