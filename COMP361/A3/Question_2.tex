\newpage
\section{Chord method to compute the cube root of 5}%
\label{sec:chord_method_to_compute_the_cube_root_of_5}

\textbf{Problem:}
Also use the Chord method to compute the cube root of 5.
Numerically carry out the first 10 iterations of the Chord method, using $x^{0} = 1$. 
Analytically determine the fixed points of the Chord iteration and determine whether they are attracting or repelling.
If a fixed point is attracting then determine analytically if the convergence is linear or quadratic.
If the convergence is linear then determine analytically the rate of convergence.
Draw the “$x^{k+1}$ versus $x^{k}$ diagram”, as in the Lecture Notes, again taking $x^{0} = 1$, and draw enough iterations in the diagram, so that the long time behavior is clearly visible.
(If done by hand then make sure that your diagram is sufficiently accurate, for otherwise the graphical results may be misleading.)
\\
Do the same computations and analysis for the Chord Method when $x^{0} = 0.1$.
\\
More generally, analytically determine all values of $x^{0}$ for which the Chord method will converge to the cube root of 5.
\\
\textbf{Solution:}
\subsection{With $x_{(0)} = 1$}

\subsubsection{Numerically carry out first 10 iterations}
Check out the full source code and presentation in directory: \textit{program/Problem2.ipynb}
Below is the main algorithm:

\begin{lstlisting}
# Main algorithm
def chord(f, diff, init_x, max_iter=1000):
    x = init_x
    estimates = []
    listX = [x]
    for i in range(max_iter):
        deltaX = -f(x)/diff(init_x)
        x = x + deltaX
        listX.append(x)
        estimates.append(x)
    return x, listX, estimates
\end{lstlisting}

The results are as following:
\begin{quote}
Xi =  1
X(i+1) =  2.333333333333333 

Xi =  2.333333333333333
X(i+1) =  -0.23456790123456672 

Xi =  -0.23456790123456672
X(i+1) =  1.4364009049609154 

Xi =  1.4364009049609154
X(i+1) =  2.115184017622358 

Xi =  2.115184017622358
X(i+1) =  0.6274038354390186 

Xi =  0.6274038354390186
X(i+1) =  2.2117476794083464 

Xi =  2.2117476794083464
X(i+1) =  0.271918086443556 

Xi =  0.271918086443556
X(i+1) =  1.9318829289112252 

Xi =  1.9318829289112252
X(i+1) =  1.1951766954547978 

Xi =  1.1951766954547978
X(i+1) =  2.2927610409500208 
\end{quote}

\subsubsection{Determine the fixed points}
\textbf{Cube root of 5} is a zero of function g (x)
Thus, let $g(x) = x^{3} - 5$

\newtheorem{mydef}{Definition}
\begin{mydef}
\textbf{Chord method}
$x^{(k+1)} = x^{k} - \frac{g(x^{k})}{g'(x^{(0)})}$
\end{mydef}

Apply Chord method with g (x) and $x_0 = 1$:
\begin{align*}
    f(x) = x - \frac{x^{3} - 5}{3x_{0}^{2}}
    = x - \frac{x^{3} - 5}{3}
    = \frac{3x - x^{3} + 5}{3}
\end{align*}
To find the fixed point of f (x), let x = f (x)
\begin{align*}
    x = f(x) \\
    \implies x = \frac{3x - x^{3} + 5}{3} \\
    \implies 3x = 3x - x^{3} + 5 \\
    \implies 5 - x^{3} = 0 \\
    \implies x = \sqrt[3]{5}
\end{align*}
The fixed point of f (x) is $x = \sqrt[3]{5}$

\subsubsection{Analyze fixed points of Chord iteration:}
\begin{align*}
    f'(x) = (x - \frac{x^{3} - 5}{3})'\\
    = 1 - x^{2}\\
    \implies |f'(x_{*})| = |1 - x_{*}^{2}| = |1 - (\sqrt[3]{5})^{2}| = 1.924 > 1
\end{align*}
$|f'(x_{*})| > 1$; thus the fixed point is \textbf{diverge} 

\subsubsection{Fixed point iteration diagram}
% Graph of the iteration no 1 \to  10
\begin{figure}[H]
\centering
\begin{tikzpicture}
   \begin{axis}[my style,restrict y to domain=-5:5,width=\linewidth, xmin=-2, xmax=4,ymin=-2,ymax=4]
      \addplot[samples=200]{(3*x-x^3+5)/(3)};
      \addplot[domain=-4:4,color=blue]{x};
      \addplot[mark=none,color=red,very thick] coordinates {
          (1,0) (1, 7/3) (7/3, 7/3) (7/3, -19/81) 
          (-19/81, -19/81) (-19/81, 1.436400905) 
          (1.436400905, 1.436400905) (1.436400905, 2.115184017622358) 
          (2.115184017622358 , 2.115184017622358 ) (2.115184017622358 , 0.6274038354390186 ) 
          (0.6274038354390186 , 0.6274038354390186 ) (0.6274038354390186 , 2.2117476794083464 )
          (2.2117476794083464 , 2.2117476794083464 ) (2.2117476794083464 , 0.271918086443556 )
          (0.271918086443556 , 0.271918086443556 ) (0.271918086443556 , 1.9318829289112252 )
          (1.9318829289112252 , 1.9318829289112252 ) (1.9318829289112252 , 1.1951766954547978 )
          (1.1951766954547978 , 1.1951766954547978 ) (1.1951766954547978 , 2.2927610409500208 )
        };
   \end{axis}
\end{tikzpicture}
\caption{Fixed point iteration no.1 to no. 10 diagram}
\end{figure}

\subsection{With $x^{0} = 0.1$}

\subsubsection{Numerically carry out first 10 iterations}

Check out the full source code and presentation in directory: \textit{program/Problem2.ipynb}

The results are as following:
\begin{quote}
Iteration no.  1
Xi =  0.1
X(i+1) =  166.7333333333333 

Iteration no.  2
Xi =  166.7333333333333
X(i+1) =  -154505913.52345666 

Iteration no.  3
Xi =  -154505913.52345666
X(i+1) =  1.229459037686188e+26 

Iteration no.  4
Xi =  1.229459037686188e+26
X(i+1) =  -6.194709380101413e+79 

Iteration no.  5
Xi =  -6.194709380101413e+79
X(i+1) =  7.923946873048754e+240 

\end{quote}
\textit{Note:} More than iteration 5, the result is too big to be computed and displayed

\subsubsection{Determine the fixed points}
Let $g(x) = x^{3} - 5$

Apply Chord method with g (x) and $x_0 = 0.1$:
\begin{align*}
    f(x) = x - \frac{x^{3} - 5}{3x_{0}^{2}}
    = x - \frac{x^{3} - 5}{0.03}
    = \frac{0.03x - x^{3} + 5}{0.03}
\end{align*}
To find the fixed point of f (x), let x = f (x)
\begin{align*}
    x = f(x) \\
    \implies x = \frac{0.03x - x^{3} + 5}{0.03} \\
    \implies 0.03x = 0.03x - x^{3} + 5 \\
    \implies x^{3} - 5 = 0 \\
    \implies x = \sqrt[3]{5}
\end{align*}
The fixed point of f (x) is $x = \sqrt[3]{5}$

\subsubsection{Analyze fixed points of Chord iteration:}
\begin{align*}
    f'(x) = (x - \frac{x^{3} - 5}{0.03})'\\
    ||f'(x_{*})| = |1- 100x_{*}^2| = |1 - 100(\sqrt[3]{5})^{2}|
    = 281.401 > 1
\end{align*}
$|f'(x_{*})| > 1$; thus the fixed point is \textbf{diverge} 

\subsubsection{Fixed point iteration diagram}
\textit{The result is too big that it is insufficient to draw such iteration graph}

\subsection{Analyze the condition of $x^{(0)}$ to make the Chord method converge}
To find the fixed point, the equation is:
\begin{align*}
    x = x - \frac{x^{3} - 5}{3x_0^2} \\
    \implies x^{3} - 5 = 0
    \implies x = \sqrt[3]{5}
\end{align*}
\textbf{Conclude:} the value of the fixed point does not depend on the value of $x_0$ in Chord method. It is always $\sqrt[3]{5}$ \\
The Chord method converges when
\begin{align*}
    \implies |f'(x_(*))| < 1
    \implies |(x - \frac{x^{3} - 5}{3x_0^2})'| < 1  \\
    \implies |1 - \frac{1}{3x_0^2}(3x^2)| < 1 
    \implies |1 - \frac{x^2}{x_0^2}| < 1 \textbf{(1)}\\
    \frac{x^2}{x_0^2} \ge 0 \forall x_0 \neq 0 \textbf{(2)}\\
    \text{From (1) and (2)}\implies 0 \le \frac{x^{2}}{x_0^2} < 1 \\
    \implies 0 \le \frac{(\sqrt[3]{5})^2}{x_0^2} < 1 
    \implies \frac{\sqrt[3]{5}^{2}}{x_0^2} - 1 < 0 \\
\implies \frac{\sqrt[3]{5}^{2}-x_0^2}{x_0^2} < 0 \text{ However, } x_0^2 > 0 \forall x_0 \neq 0 \\
    \implies \sqrt[3]{5}^2 - x_0^2 < 0
    \implies x_0^2 > \sqrt[3]{5}^2
    \implies x_0 > \sqrt[3]{5}
\end{align*}
\textbf{Conclusion:} The Chord method converges only when $x_0 > \sqrt[3]{5}$
\newpage
