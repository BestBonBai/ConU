\newpage

\section{Using Newton's method to find $\sqrt(R)$ }%
\label{sec:newton_s_method_to_find_sqrt_r_}

\textbf{Problem:}
Newton's method to find $\sqrt(R)$ is:\\
\\
$x_{n+1} = \frac{1}{2} (x_n + \frac{R}{x_n})$ \\
\\
\begin{itemize}
    \item Perform three iterations of scheme (1) for computing $\sqrt(2)$, starting with $x_0 = 1$.
    \item Perform three iterations of the bisection method for computing $\sqrt(2)$, starting with interval [1,2].
    \item Find theoretically the minimum number of iterations in both schemes to achieve $10^{-6}$ accuracy.
    \item Find numerically the minimum number of iterations in both schemes to achieve $10^{-6}$ accuracy and compare your results with the theoretical estimates.
\end{itemize}

\textbf{Solution:}
\subsubsection{Iterations using Newton scheme}
The iterations were conducted using Jupyter Notebook with Python. The main algorithm is as followed. The full source code can be found at \textit{program/Problem5.ipynb}
\begin{lstlisting}
def newton_f(x, R):
    return (1/2)*(x + (R/x))

def newton_method(f, x, R, NumOfIteration = 1000):
    init_x = x
    estimates = []
    listX = [x]
    for i in range(numOfIteration):
        x = newton_f(x, R)
        listX.append(x)
        estimates.append(x)
    return listX, estimates
\end{lstlisting}

The results are as following:
\begin{quote}
Iteration no. 1
X =  1
Xi =  1.5

Iteration no. 2
X =  1.5
Xi =  1.4166666666666665

Iteration no. 3
X =  1.4166666666666665
Xi =  1.4142156862745097

Final estimation:  1.4142156862745097
\end{quote}

\subsubsection{Iterations using Bisection method}
The iterations were conducted using Jupyter Notebook with Python. The main algorithm is as followed. The full source code can be found at \textit{program/Problem5.ipynb}
\begin{lstlisting}
def f(x):
    return x**2 -2 

def bisection_method(f, a, b, NumOfIteration = 1000):
    """
    a, b: the interval
    f: the function to be approximated
    """
    if f(a) * f(b) >= 0:
        print('Bisection method fails')
        return None, None
    
    a_n = a
    b_n = b
    midpointRecord = []
    for n in range(0, NumOfIteration):
        midpoint = (a_n + b_n)/2
        midpointRecord.append(midpoint)
        fMidpoint = f(midpoint)
        if f(a_n) * fMidpoint < 0:
            b_n = midpoint
        elif f(b_n) * fMidpoint <0:
            a_n = midpoint
        elif fMidpoint == 0:
            print('Found exact solution')
            return midpoint
    return (a_n + b_n)/2, midpointRecord
\end{lstlisting}

The results are as following:
\begin{quote}
Iteration no. 1
Midpoint:  1.5

Iteration no. 2
Midpoint:  1.25

Iteration no. 3
Midpoint:  1.375

Final estimation:  1.4375
\end{quote}

% Graph of the iteration
\begin{figure}[H]
\centering
\begin{tikzpicture}
   \begin{axis}[my style,restrict y to domain=-5:5,width=\linewidth, xmin=-2, xmax=4,ymin=-2,ymax=4]
      \addplot[samples=200]{x^2 - 2};
      \addplot[domain=-4:4,color=blue]{x};
      \addplot[mark=none,color=red,very thick] coordinates {
          (1, -1)(1, 0)
      };
      \addplot[mark=none,color=red,very thick] coordinates {
          (2, 2)(2, 0)
      };
   \end{axis}
\end{tikzpicture}
\caption{Bisection method demonstration diagram}
\end{figure}
\newpage

\subsubsection{Minimum number of iterations for $10^{-6}$ accuracy}
\textbf{Scheme (1)}\\
\textbf{Bisection method} \\
Error of the approximation: 
\[
    e_{n} = |r - c_{n}| \le \frac{b_{0} - a_{0}}{2}\frac{1}{2^{n}}
.\] 
With
\begin{itemize}
    \item n: the $n^{th}$ iteration
    \item $e_{n}$ Denotes the error
    \item r: the real value
    \item  $c_{n}$: The estimation at iteration n
    \item  $a_{0}, b_{0}$: The 2 starting points of the starting interval
\end{itemize}

\begin{align*}
    \implies e_{n} \le \frac{2-1}{2} \frac{1}{2^{n}} \le 10^{-6} \\
    \implies 2^{n+1} \ge 10^{6}
    \implies n + 1 \ge 20
    \implies n \ge  19
\end{align*}
\textbf{Conclusion:} With the minimum of number of iteration is 19, the function will have $10^{-6}$ accuracy.




\newpage

