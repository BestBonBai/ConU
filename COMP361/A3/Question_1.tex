\newpage
\section{Newton's method to compute the cube root of 6}
\label{sec:newton_s_method_to_compute_the_cube_root_of_6}

\textbf{Problem:}
Show how to use Newton’s method to compute the cube root
of 5. Numerically carry out the first 10 iterations of Newton’s method, using $x_{0}$ = 1. 
Analytically determine the fixed points of the Newton iteration and determine whether they are attracting or repelling.
If a fixed point is attracting then determine analytically if the convergence is linear or quadratic.
Draw the “$x_{k+1}$ versus $x_{k}$ diagram”, again taking $x_{0} = 1$, and draw enough iterations in the diagram, so that the long time behavior is clearly visible. 
For which values of $x_{0}$ will Newton’s method converge?
\\
\textbf{Solution:}
\subsection{Iterations using Newton's method}
\label{10iterations}
Cube root of 5 is a zero of the function g (x) such that:\\
$g(x) = x^{3} - 5$\\

The first 10 iterations using Newton's method were carried out by using Python with Jupyter Notebook. 
Check out the full source code and presentation in directory: \textit{program/Problem1.ipynb}
Below is the main algorithm:
\begin{lstlisting}
# The main algorithm
def newton_raphson(f, diff, init_x, max_iter=1000):
    x = init_x
    estimates = []
    listX = [x]
    for i in range(max_iter):
        deltaX = -f(x)/diff(x)
        x = x + deltaX
        listX.append(x)
        estimates.append(x)
    return x, listX, estimates
    
\end{lstlisting}

The results after computing are as following:
\begin{quote}
Iteration no.  1
Xi =  1
X(i+1) =  2.333333333333333 

Iteration no.  2
Xi =  2.333333333333333
X(i+1) =  1.8616780045351473 

Iteration no.  3
Xi =  1.8616780045351473
X(i+1) =  1.722001880058607 

Iteration no.  4
Xi =  1.722001880058607
X(i+1) =  1.7100597366002945 

Iteration no.  5
Xi =  1.7100597366002945
X(i+1) =  1.709975950782189 

Iteration no.  6
Xi =  1.709975950782189
X(i+1) =  1.709975946676697 

Iteration no.  7
Xi =  1.709975946676697
X(i+1) =  1.709975946676697 

Iteration no.  8
Xi =  1.709975946676697
X(i+1) =  1.709975946676697 

Iteration no.  9
Xi =  1.709975946676697
X(i+1) =  1.709975946676697 

Iteration no.  10
Xi =  1.709975946676697
X(i+1) =  1.709975946676697 
\end{quote}

\subsection{Determine fixed points}
\label{Fixed point finding}
\textbf{Newton's method:}
\begin{centering}
    $x_{n+1} = x_{n} - \frac{g(x_{n})}{g'(x_{n})}$
\end{centering}

Applying Newton's method\\
\begin{align*}
\implies f(x) = x - \frac{x^{3} - 5}{3x^{2}} = \frac{3x^{3} - x^{3} + 5}{3x^{2}}\\
= \frac{2x^{3} + 5}{3x^{2}} 
\end{align*}

To find the fixed points, let x = f (x)\\
\begin{align*}
x = \frac{2x^{3} + 5}{3x^{2}} \\
3x^{3} = 2x^{3} + 5  \\
x^{3} = 5 \\
\implies x = \sqrt[3]{5}
\end{align*}
The fixed point of the Newton iteration is $x= \sqrt[3]{5}$ \\

\subsection{Fixed point analytically}
\begin{align*}
    f'(x) = (\frac{2x^{3} + 5}{3x^{2}})'
    = (\frac{2}{3}x + \frac{5}{3x^{2}})'
    = \frac{2}{3} - \frac{10}{3x^{3}} \\
    \\
    \implies|f'(x^{*})| = |f'(\sqrt[3]{5})| = |\frac{2}{3} - \frac{10}{3(\sqrt[3]{5})^{3}}| = |\frac{2}{3} - \frac{10}{15}| = 0
\end{align*}
$|f'(x^{*})| = 0 \implies$ the fixed point is \textbf{attracting} and it is converging \textbf{quadratically}.

\subsection{Fixed point iteration diagram}
% Graph of the iteration
\begin{figure}[H]
\centering
\begin{tikzpicture}
   \begin{axis}[my style,restrict y to domain=-5:5,width=\linewidth, xmin=-2, xmax=4,ymin=-2,ymax=4]
      \addplot[samples=200]{(2*x^3+5)/(3*x^2)};
      \addplot[domain=-4:4,color=blue]{x};
      \addplot[mark=none,color=red,very thick] coordinates {
          (1,0) (1, 7/3) (7/3, 7/3) (7/3, 821/441) (821/441, 821/441) (821/441, 1.722) (1.722, 1.722) (1.722, 1.710) (1.710, 1.710) (1.710, 7099)
      };
   \end{axis}
\end{tikzpicture}
\caption{Fixed point iteration diagram}
\end{figure}

\subsection{Newton's method converges}
The fixed point is always attracting regardless of the value of $x^{0}$; thus the Newton's method always converges at the fixed point.

\newpage
