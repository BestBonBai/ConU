\newpage
\section{Solving a system of nonlinear equations by Newton's method}%
\label{sec:solving_a_system_of_nonlinear_equations_by_newton_s_method}

\textbf{Problem}
\\
Consider the example of solving a system of nonlinear equations by Newton’s method, as given on Pages 95 to 97 of the Lecture notes. 
Write a program to carry out this iteration, using Gauss elimination to solve the 2 by 2 linear systems
that arise.
Use each of the following 16 initial data sets for the Newton iteration:
\\
\\
($x_{1}^{0}, x_{2}^{0}$) = (i, j),\\
$i = 0, 1, 2, 3, $\\
 $j = 0, 1, 2, 3$.\\
 \\
Present and discuss your numerical results in a concise manner.
\\
\textbf{Solution:}

The code and full presentation is available in the directory: \textit{program/Problem4.ipynb} \\

Below is the supporting functions for the main algorithm:
\begin{lstlisting}
# The original system of linear equation
def G(x):
    g1 = pow(x[0], 2)*x[1] - 1
    g2 = x[1] - pow(x[0], 4)
    return np.array([g1, g2], dtype = np.float)

def convertG(x):
    for number in x:
        if(number != 0):
            number = -1 * number
    return x

# Jacobian matrix
def dG(x):
    J = np.zeros([len(x), len(x)], dtype = np.float)
    eps = 1e-10
    for i in range(len(x)):
        x1 = x.copy()
        x2 = x.copy()

        x1[i] += eps
        x2[i] -= eps

        g1 = G(x1)
        g2 = G(x2)

        J[ : ,i] = (g1 - g2) / (2 * eps)
    
    return J

def calculateDeltaX(dG, G):
    return np.linalg.solve(dG, G)

def estimateIteration(guess, deltaX):
    return np.add(guess, deltaX)
\end{lstlisting}

The main functions are as following: 
\begin{lstlisting}
def solveIteration(guessX1, guessX2):
    initGuess = np.array([guessX1, guessX2], dtype = np.float)
    funcG = G(initGuess)
    funcG = convertG(funcG)
    funcDG = dG(initGuess)
    deltaX = calculateDeltaX(funcDG, funcG)
    results = estimateIteration(initGuess, deltaX)
    return results

def SolveDataSet(rangeSet):
    count = 0
    for i in guessSet:
        for j in guessSet:
            count = count + 1
            print("\nIteration: no.", count)
            print("i = ", i, " j = ", j)
            estimation = solveIteration(i, j)
            print("x1 = ", estimation[0], " x2 = ", estimation[1])

guessSet = range(1, 4)
SolveDataSet(guessSet)
\end{lstlisting}
The output is as following:
\begin{quote}
    Iteration: no. 1
i =  1  j =  1
x1 =  1.0  x2 =  1.0

Iteration: no. 2
i =  1  j =  2
x1 =  1.0  x2 =  2.999999917259636

Iteration: no. 3
i =  1  j =  3
x1 =  1.0  x2 =  4.999999834519272

Iteration: no. 4
i =  1  j =  4
x1 =  1.0  x2 =  6.999999751778907

Iteration: no. 5
i =  2  j =  1
x1 =  2.477272687783008  x2 =  1.2727272501617188

Iteration: no. 6
i =  2  j =  2
x1 =  2.46323525578939  x2 =  2.8235293436255824

Iteration: no. 7
i =  2  j =  3
x1 =  2.449999962766836  x2 =  4.39999988416349

Iteration: no. 8
i =  2  j =  4
x1 =  2.437499963801091  x2 =  5.999999834519272

Iteration: no. 9
i =  3  j =  1
x1 =  3.7443757939848985  x2 =  1.3926382860186328

Iteration: no. 10
i =  3  j =  2
x1 =  3.7398369866787884  x2 =  2.9024376358618897

Iteraton: no. 11
i =  3  j =  3
x1 =  3.7353532030813827  x2 =  4.418179444595575

Iteration: no. 12
i =  3  j =  4
x1 =  3.730923441770851  x2 =  5.939755560512124

Iteration: no. 13
i =  4  j =  1
x1 =  4.997806722363986  x2 =  1.4385965612489156

Iteration: no. 14
i =  4  j =  2
x1 =  4.995865220763078  x2 =  2.9416346189274662

Iteration: no. 15
i =  4  j =  3
x1 =  4.993931258042135  x2 =  4.446602869886977

Iteration: no. 16
i =  4  j =  4
x1 =  4.992004790375868  x2 =  5.95349009345808
\end{quote}

\newpage

