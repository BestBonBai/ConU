\documentclass[a4paper]{article}

\usepackage{listings}
\usepackage{xcolor}
\usepackage[T1]{fontenc}
\usepackage{textcomp}
\usepackage{amsthm}
\usepackage{amsmath, amssymb}
\usepackage{enumitem}
\usepackage{transparent}
\usepackage[outline]{contour}
\usepackage[affil-it]{authblk}

\title{COMP 353 Databases\\
Assignment no.4}

\author{Duc Nguyen}
 \affil{Gina Cody School of Computer Science and Software Engineering \\
    Concordia University, Montreal, QC, Canada}
\date{Summer 2020}

\begin{document}
\maketitle

\newpage
\tableofcontents
\newpage

\section{Problem description:}
Consider a DB schema consisting of the following relation schemes:
\begin{itemize}
    \item \textbf{Regions} (\underline{\textbf{Region\_ID}}, Region\_Name)
    \item \textbf{Countries} (\underline{\textbf{Country\_id}}, Country\_Name, Region\_Id)
    \item \textbf{Locations} (\underline{\textbf{Location\_Id}}, Street\_address, Postal\_code, City, State\_Province, Country\_Id)
    \item \textbf{Jobs} (\underline{\textbf{Job\_Id}},  Job\_title, Min\_Salary, Max\_salary)
    \item \textbf{Departments} (\underline{\textbf{Dep\_Id}},  Dep\_Name, Manager\_Id,
Location\_Id)
    \item \textbf{Employees} (\underline{\textbf{Emp\_ID}},  FirstName, Last\_Name, E-mail,
Phone\_number, Hire\_date, Job\_Id, Salary, Comsn\_pct,
Manager\_Id, Dep\_Id)
    \item \textbf{Employee\_History} (\underline{\textbf{Emp\_ID}, \textbf{Joining\_date}}, last\_date, Job\_ID, Dep\_ID)
\end{itemize}

\textbf{Keys are underlined} \\
Now, express the following queries in SQL:
\newpage
\subsection{Query 1.}
\subsubsection{Description}
Find the count of departments, region name(s) and cities for the department(s)
that have more than 500 employees.
\subsubsection{SQL code}
\begin{lstlisting}[language=SQL]
select count(Dep_Id), Region_Name, City
from Regions R
join Countries C on R.Region_ID = C.Region_ID 
join Locations L on C.Country_Id = L.Country_Id 
join Departments D on D.Location_Id = L.Location_Id
where Dep_Id in (
    select Dep_Id 
    from Employees
    group by Dep_Id 
    having count(Dep_Id) > 500
)
group by Region_Name, City ;
\end{lstlisting}

\subsection{Query 2.}
\subsubsection{Description}
For a department in which the max salary is greater than 100000 for employees
who worked in the past, set the manager name as ‘Picard’
\begin{lstlisting}[language=SQL]
update Employees e
set FirstName = 'Picard' 
where Dep_Id in 
    (select  Dep_Id 
    from Employees
    where Emp_ID in 
        (select Emp_ID from Employee_History )
    group by Dep_Id
    having max(Salary) > 10000);
\end{lstlisting}

\newpage
\subsection{Query 3.}
\subsubsection{Description}
Find month and year which witnessed lowest count of employees joining a
department located in ‘Vancouver’.
\subsubsection{SQL Query}
\begin{lstlisting}[language=SQL]
select month(Joining_date), year(Joining_date)
from Employee_History E
join Departments D on E.Dep_Id = D.Dep_Id
join Locations L on D.Location_Id = L.Location_Id
where L.City = 'Vancouver'
having count(E.Emp_ID) = (
    select min(count(E.Emp_ID))
    from Employee_History H
    join Departments on H.Dep_Id = D.Location_Id
    join Locations on D.Location_Id = L.Location_Id
    where L.City = 'Vancouver'
);
\end{lstlisting}


\subsection{Query 4.}
\subsubsection{Description}
With the help of schema find the year which witnessed maximum number of
employee intake.
\subsubsection{SQL Query}
\begin{lstlisting}[language=SQL]
select year(A.Joining_date)
from Employee_History A
join Employees B
on A.Emp_ID=B.Emp_ID
group by year(A.Joining_date)
group by count(*) desc
limit 1;
\end{lstlisting}

\newpage
\subsection{Query 5.}
\subsubsection{Description}
For the year in query-4, find how many joined in each month in that specific year
\subsubsection{SQL Query}
\begin{lstlisting}[language=SQL]
select count(Emp_ID), month(Joining_date)
from Employee_History h
where year(Joining_date) = (
	select year(A.Joining_date)
	from Employee_History A
	join Employees B
	on A.Emp_ID=B.Emp_ID
	group by year(A.Joining_date)
	group by count(*) desc
	limit 1
)
group by month(Joining_date)
order by count(Emp_ID);
\end{lstlisting}

\newpage
\textbf{PART 2: Use triggers for the queries below:}
\subsection{Query 6.}
\subsubsection{Description}
Create a trigger to ensure that a salary of an employee cannot exceed the salary of
his/her manager. If the employee does not have a manager, then his/her salary
cannot be more than 10\% of the highest salary in the database.
\subsubsection{SQL Query}
\begin{lstlisting}
CREATE TRIGGER A
BEFORE INSERT OR UPDATE OF Salary, Manager_Id ON Employees
NEW ROW AS new
FOR EACH ROW
WHEN (new.Salary > (SELECT Salary
    FROM Employees
    WHERE Emp_ID = Manager_Id))
    OR 
    ( new.Emp_ID IN (SELECT Emp_ID FROM Employees WHERE Manager_Id = NULL)
    AND new.Salary > 1.1 * ( SELECT MAX(Salary) FROM Employees)
Begin
    ROLLBACK;
End;
\end{lstlisting}

\newpage
\subsection{Query 7.}
\subsubsection{Description}
For changes in the job of an employee, updated details provided below must be
written to Employee History:
hire date of the employee for start date, old job ID, old department ID,
Employee ID, todays' system date for end date. In case a row is already
present for employee job history then the start date must be the end date of that
(row +1).
\subsubsection{SQL Query}
\begin{lstlisting}
CREATE TRIGGER B
AFTER INSERT OR UPDATE OF Job_Id ON Employees
FOR EACH ROW
referencing
    old row as Old
    new row as New
declare
    enddate date;
    startdate date;
begin
    select max(last_date) into enddate
    from Employee_History
    where Employee_History.Emp_ID = old.Emp_ID;
    
    if enddate is null:
        insert into 
        Employee_History (Joining_date, Job_ID, Dep_ID, Emp_ID, last_date) 
        values (Old.Hire_date, Old.Job_Id, Old.Dep_Id, Old.Emp_ID, sysdate);
    else:
        startdate = enddate + 1;
        update Employee_History
        set Joining_date = startdate, last_date = sysdate
        where Old.Emp_ID = Employee_History.Emp_ID;
    end if;
\end{lstlisting}

\newpage
\subsection{Query 8.}
\subsubsection{Description}
Make a Trigger to ensure that the salary of the employee is never decreased while
working in an organization.
\subsubsection{SQL Query}
\begin{lstlisting}
create trigger C
after update of Salary on Employees
referencing old row as OldT, new row as NewT
for each row
when (OldT.Salary > NewT.Salary)
begin
    update Employees
    set Salary = OldT.Salary
end;
\end{lstlisting}


\subsection{Query 9.}
\subsubsection{Description}
Create a trigger to ensure that an increase of salary for an employee is conform
with the following rules: If experience is more than 8 years, increase salary by
max 20\%; If experience is greater than 3 years, increase salary by max of 10\%;
Otherwise a max increase of 5\%.
\subsubsection{SQL Query}
\begin{lstlisting}
create trigger D
before update of Salary on Employees
referencing 
    old row as Old
    new row as New
for each row
when (
    ((getdate() - new.Hire_date) < 8 and (new.Salary > 1.20 * old.Salary) )
    or ((getDate() - new.Hire_date) < 3 and (new.Salary > 1.10 * old.Salary))
    or (new.Salary > 1.05 * old.Salary)
    )
begin
    rollback;
end;
\end{lstlisting}

\newpage
\subsection{Query 10.}
\subsubsection{Description}
Create a trigger to ensure that Min\_salary cannot exceed Max\_salary for any job
\subsubsection{SQL Query}
\begin{lstlisting}
create trigger E
before update on Jobs
referencing 
    new row as newTuple
for each row
when (newTuple.Min_Salary > newTuple.Max_Salary)
begin 
    rollback;
end 
\end{lstlisting}

\end{document}
